%
% Exemplo LaTeX de monografia UNISINOS
%
% Elaborado com base nas orientações dadas no documento
% ``GUIA PARA ELABORAÇÃO DE TRABALHOS ACADÊMICOS''
% disponível no site da biblioteca da Unisinos.
% http://www.unisinos.br/biblioteca
%
% Os elementos textuais abaixo são apresentados na ordem em que devem
% aparecer no documento.  Repare que nem todos são obrigatórios - isso
% é devidamente indicado em cada caso.
%
% Comentários abaixo colocados entre aspas (`` '') foram
% extraídos diretamente do documento da biblioteca.
%
% Este documento é de domínio público.
%

%=======================================================================
% Declarações iniciais identificando a classe de documento e
% selecionando alguns pacotes adicionais.
%
% As opções disponíveis (separe-as com vírgulas, sem espaço) são:
% - twoside: Formata o documento para impressão frente-e-verso
%   (o default é somente-frente)
% - english,brazilian,french,german,etc.: idiomas usados no documento.
%   Deve ser colocado por último o idioma principal.
%=======================================================================
\documentclass[twoside,brazilian,english]{UNISINOSmonografia}
\usepackage[utf8]{inputenc} % charset do texto (utf8, latin1, etc.)
\usepackage[T1]{fontenc} % encoding da fonte (afeta a sep. de sílabas)
\usepackage{graphicx} % comandos para gráficos e inclusão de figuras
\usepackage{bibentry} % para inserir refs. bib. no meio do texto
\usepackage{colortbl}
\usepackage{multirow}



%=======================================================================
% Escolha do sistema para geração de referências bibliográficas.
%
% O default é usar o estilo unisinos.bst.  Comente a definição abaixo
% e descomente a linha seguinte para usar o estilo do ABNTeX (é
% necessário ter esse pacote instalado).
%
% A vantagem do unisinos.bst é que ele permite o uso de um arquivo .bib
% seguindo as orientações tradicionais do BibTeX (veja essas orientações
% em http://ctan.tug.org/tex-archive/biblio/bibtex/contrib/doc/btxdoc.pdf).
% Entretanto, o estilo não suporta algumas citações mais exóticas como
% apud.  Para isso, use o ABNTeX, mas esteja ciente de que muitas de
% suas referências serão incompatíveis com os estilos tradicionais do
% BibTeX como plain, alpha, ieeetr, entre outros.
%=======================================================================
\usepackage[alf]{abntex2cite}
%\usepackage[alf]{abntcite}

%=======================================================================
% Dados gerais sobre o trabalho.
%=======================================================================
\autor{Uhlein}{Gustavo}
\titulo{Security and Defense Mechanisms in AI Systems}
\subtitulo{A Systematic Review}
\orientador[Master of Science (M.Sc.)]{Goldoni}{Diógines}
%\coorientador[Prof.~Dr.]{Lamport}{Leslie}

\unidade{Unidade Acadêmica de Graduação}
\curso{Curso de Bacharelado em Ciência da Computação}
\natureza{%
Monografia apresentada como requisito parcial para obtenção do título de Bacharel em Ciência da Computação, pelo Curso de Ciência da Computação da Universidade do Vale do Rio dos Sinos (UNISINOS)
}
\local{Porto Alegre}
\ano{2025}

% dados da ficha catalográfica
% (obrigatória somente para dissertações e teses)
%\cip{Dissertação (mestrado)}{004.732}
%\bibliotecario{Bibliotecária responsável: Fulana da Silva}{12/3456}

% cada palavra-chave deve ser fornecida duas vezes, uma em português e
% outra no idioma estrangeiro (na verdade, em tantos idiomas quantos se
% desejar).
\palavrachave{brazilian}{Segurança em Inteligência Artificial}
\palavrachave{brazilian}{Ataques a Sistemas de IA}
\palavrachave{brazilian}{Mecanismos de Defesa em IA}
\palavrachave{brazilian}{Vetores de Ataque}
\palavrachave{brazilian}{Ataques Adversariais}
\palavrachave{brazilian}{Envenenamento de Dados}
\palavrachave{brazilian}{Inversão de Modelos}
\palavrachave{brazilian}{Exfiltração de Dados}
\palavrachave{brazilian}{Engano do Modelo}
\palavrachave{brazilian}{Desestabilização de Sistemas}
\palavrachave{brazilian}{Tendências em Publicações}
\palavrachave{brazilian}{Revisão Sistemática}
\palavrachave{english}{Artificial Intelligence Security}
\palavrachave{english}{Attacks on AI Systems}
\palavrachave{english}{Defense Mechanisms in AI}
\palavrachave{english}{Attack Vectors}
\palavrachave{english}{Adversarial Attacks}
\palavrachave{english}{Data Poisoning}
\palavrachave{english}{Model Inversion}
\palavrachave{english}{Data Exfiltration}
\palavrachave{english}{Model Evasion}
\palavrachave{english}{System Destabilization}
\palavrachave{english}{Publication Trends}
\palavrachave{english}{Systematic Review}

%=======================================================================
% Início do documento.
%=======================================================================
\begin{document}
\capa
\folhaderosto
%\folhadeaprovacao % não deve ser incluída nos TCCs

%=======================================================================
% Dedicatória (opcional).
%
% O texto é normalmente colocado na parte de baixo da página, alinhado
% à direita.  Mas a formatação é basicamente livre.  Só não se escreve
% a palavra 'dedicatória'.
%=======================================================================
% \begin{dedicatoria}
%Aos nossos pais.\\[4ex] % quebra a linha dando um espaçamento maior
% \begin{itshape} % faz o texto ficar em itálico
%If I have seen farther than others,\\
% it is because I stood on the shoulders of giants.\\
% \end{itshape}
% --- \textsc{Sir Isaac Newton} % \textsc é o "small caps"
% \end{dedicatoria}

%=======================================================================
% Agradecimentos (opcional).
%=======================================================================
% \begin{agradecimentos}
% Obrigado!
% \end{agradecimentos}

%=======================================================================
% Epígrafe (opcional).
%
% ``[...] o autor apresenta uma citação, seguida de indicação de autoria,
% relacionada com a matéria tratada no corpo do trabalho. Podem, também,
% constar epígrafes nas folhas de aberturas das seções primárias.''
%=======================================================================
% \begin{epigrafe}
% ``\textit{Ninguém abre um livro sem que aprenda alguma coisa}''.\\
% (Anônimo)
% \end{epigrafe}

%=======================================================================
% Resumo.
%
% A recomendação é para 150 a 500 palavras.
%=======================================================================
\begin{abstract}
This systematic review investigates the landscape of security and defense mechanisms in Artificial Intelligence (AI) systems, focusing on the most prevalent types of attacks, their objectives, proposed defenses, evaluation metrics, and publication trends between 2020 and 2025. Guided by five research questions, the study synthesizes findings from peer-reviewed primary sources published in English, following strict inclusion and exclusion criteria to ensure academic relevance and rigor. The review identifies the most studied attack vectors against AI systems—such as adversarial attacks, data poisoning, and model inversion—along with their typical objectives, including data exfiltration, model deception, and system destabilization. Additionally, it explores the defense mechanisms proposed in the literature, analyzing how they align with the identified attack types and their effectiveness in mitigating vulnerabilities. The study also examines the metrics used to evaluate both attacks and defenses. Finally, it assesses the publication trend over recent years, revealing a significant increase in AI security research, reflecting growing academic and practical interest. This review offers a comprehensive overview of the current state of AI security, providing valuable insights for researchers and practitioners aiming to develop robust defense strategies and address emerging threats in AI systems.
\end{abstract}

%=======================================================================
% Resumo em língua estrangeira (obrigatório somente para teses e
% dissertações).
%
% O idioma usado aqui deve necessariamente aparecer nos parâmetros do
% \documentclass, no início do documento.
%=======================================================================
\begin{otherlanguage}{brazilian}
\begin{abstract}
Esta revisão sistemática investiga o panorama dos mecanismos de segurança e defesa em sistemas de Inteligência Artificial (IA), com foco nos tipos de ataques mais prevalentes, seus objetivos, defesas propostas, métricas de avaliação e tendências de publicações entre 2020 e 2025. Através de cinco questões de pesquisa, este estudo sintetiza achados de fontes primárias revisadas por pares, publicadas em inglês, seguindo critérios rigorosos de inclusão e exclusão para garantir relevância e rigor acadêmico. A revisão identifica os vetores de ataque mais investigados contra sistemas de IA, como ataques adversariais, envenenamento de dados e inversão de modelos, juntamente com seus objetivos típicos, incluindo exfiltração de dados, engano do modelo e desestabilização do sistema. Além disso, explora os mecanismos de defesa propostos na literatura, analisando seu alinhamento com os tipos de ataques identificados e sua eficácia na mitigação de vulnerabilidades. O estudo também examina as métricas utilizadas para avaliar ataques e defesas. Por fim, avalia a tendência no volume de publicações, revelando um aumento significativo na pesquisa sobre segurança em IA, refletindo o crescente interesse acadêmico e prático. Esta revisão oferece uma visão abrangente do estado atual da segurança em IA, fornecendo insights para pesquisadores e profissionais desenvolverem estratégias de defesa robustas e enfrentarem ameaças emergentes em sistemas de IA.
\end{abstract}
\end{otherlanguage}

%=======================================================================
% Lista de Figuras (opcional).
%=======================================================================
% \listoffigures

%=======================================================================
% Lista de Tabelas (opcional).
%=======================================================================
\listoftables

%=======================================================================
% Lista de Abreviaturas (opcional).
%
% Deve ser passada como parâmetro a maior das abreviaturas utilizadas.
%=======================================================================
% \begin{listadeabreviaturas}{seg., segs.}
% \item[ampl.] ampliado, -a
% \item[atual.] atualizado, -a
% \item[coord.] coordenador
% \item[N.~T.] Novo Testamento
% \item[seg., segs.] seguinte, -s
% \end{listadeabreviaturas}

%=======================================================================
% Lista de Siglas (opcional).
%
% Deve ser passada como parâmetro a maior das siglas utilizadas.
%=======================================================================
\begin{listadesiglas}{FAPERGS}
\item[ABNT] Associação Brasileira de Normas Técnicas
\item[CAPES] Coordenação de Aperfeiçoamento de Pessoal de Nível Superior
\item[FAPERGS] Fundação de Amparo à Pesquisa do Estado do Rio Grande do Sul
\end{listadesiglas}

%=======================================================================
% Lista de Símbolos (opcional).
%
% Deve ser passado o maior (mais largo) dos símbolos utilizados.
%=======================================================================
% \begin{listadesimbolos}{Ca}
% \item[\textsuperscript{o}C] Graus Celsius
% \item[Al] Alumínio
% \item[Ca] Cálcio
% \end{listadesimbolos}

%=======================================================================
% Sumário
%=======================================================================
\tableofcontents

%=======================================================================
% Introdução
%=======================================================================
\chapter{Introduction}

The progressive integration of Artificial Intelligence (AI) into critical infrastructure domains and essential spheres of human activity has reshaped the global technological landscape \cite{Russell2022}. This transformation spans from the financial sector, with its algorithmic trading systems and fraud detection \cite{Henrique2019}, to healthcare, where it assists in complex diagnoses and treatment personalization \cite{Rajpurkar2017, Chen2020}, to autonomous vehicles, which promise to revolutionize urban mobility and logistics \cite{Grigorescu2020}, and even to cybersecurity itself, where it is used to identify and mitigate digital threats \cite{Russell2022}.

As these AI systems, characterized by increasing autonomy and the ability to make high-impact decisions with minimal human intervention, become more prevalent, their robustness, reliability, and security emerge not only as technical requirements but as central ethical, economic, and social concerns \cite{Bommasani2021}. The inherent complexity of the most advanced AI models, such as deep neural networks \cite{Hinton2006}, which often operate as 'black boxes' due to the difficulty of fully auditing and interpreting their internal decision-making processes, creates a unique and ever-expanding attack surface \cite{Papernot2018}.

This opacity makes AI models particularly vulnerable to perturbations—small modifications that are almost imperceptible to the human eye but can result in incorrect classifications with high confidence. Such perturbations not only reveal structural flaws in generalization mechanisms but also demonstrate a surprising ability to transfer between different models, even those trained with different architectures or datasets \cite{Szegedy2014, Goodfellow2015}. These findings show that current systems, although effective under normal conditions, can be led to incorrect behaviors, posing a concrete risk to applications that require reliability, robustness, and predictability in sensitive or mission-critical environments. This dynamic and challenging landscape creates a continuous arms race in the digital domain: as new, more powerful, and capable AI architectures are developed and implemented \cite{Vaswani2017, OpenAI2023}, new—often unforeseen—vulnerabilities are simultaneously discovered and exploited, demanding an incessant and proactive effort in the development and adaptation of increasingly sophisticated and resilient defense mechanisms \cite{Madry2018, Cohen2019}. In this context, the security of AI systems is established as an absolutely essential and non-negotiable pillar for the safe, responsible, and ethical adoption of these transformative technologies \cite{NIST2024}.

The threats to the security of AI systems encompass a vast and heterogeneous spectrum, extending considerably beyond the vulnerabilities traditionally associated with conventional software \cite{Anderson2020}. Attacks can manifest in multiple and distinct phases of the machine learning (ML) model lifecycle, from the initial stage of data collection and curation, through the model's learning and optimization process, to the inference phase, when the model is effectively deployed and used in production to make decisions or predictions \cite{Papernot2018}.

These attacks target a diverse set of malicious objectives, including, but not limited to, the exfiltration of sensitive and proprietary data through model extraction attacks \cite{Tramer2016}, the systematic deception of the model to induce it into erroneous classifications with potentially severe consequences through adversarial attacks \cite{Carlini2017}, the violation of privacy through membership inference attacks \cite{Shokri2017}, and the widespread destabilization of systems through data poisoning \cite{Biggio2012, Gu2017}.

A deep and detailed understanding of these attack vectors, their technical nuances, and underlying objectives is the first and indispensable step for the development and implementation of effective and robust countermeasures. Consequently, the field of AI security research is experiencing a moment of effervescence and intense activity, with the global scientific community engaged in a dual effort: on the one hand, to proactively identify and characterize emerging vulnerabilities; on the other, to propose and validate intrinsically more robust model architectures, as well as innovative and adaptive defense techniques, including adversarial training \cite{Zhang2019}, differential privacy \cite{Abadi2016}, and anomaly detection methods \cite{Ma2018}.

The domain of Artificial Intelligence security is undeniably in a state of rapid and continuous expansion, evidenced by the exponentially growing volume of scientific publications, patents, and research initiatives each year. This proliferation of knowledge, while intrinsically positive and indicative of the area's vitality and relevance, simultaneously poses a considerable challenge for researchers, developers, industry professionals, and policymakers, who strive to stay up-to-date with the latest trends, the most critical and relevant threats, and the most promising and effective defense strategies.

The relevant scientific literature is often scattered across a myriad of specialized conferences and journals from various fields (computer science, engineering, statistics, ethics), often lacking a comprehensive and systematic synthesis that organizes, categorizes, and consolidates the most recent and significant findings in an accessible and integrated manner. In this scenario, conducting a systematic literature review (SLR) is fully justified by the pressing need to map the current state of AI security research in a rigorous, transparent, and replicable way, following established methodologies \cite{Kitchenham2007, Petersen2008}.

By answering specific research questions and applying a robust methodological protocol, this study has the potential to offer valuable contributions: first, to consolidate existing knowledge, providing a panoramic, comprehensive, and structured overview of the predominant types of attacks, proposed defense mechanisms, and evaluation metrics used; second, to identify emerging and consolidated trends by analyzing the evolution of academic and industrial interest in the topic and highlighting areas of special relevance; third, to support researchers and newcomers to the field by offering a solid and well-founded starting point for future investigations, showcasing what has already been established; and finally, to guide industry professionals and developers by presenting practical insights into the most critical vulnerabilities and protection strategies that should be considered and prioritized in the development, deployment, and maintenance of secure and reliable AI systems.

%=======================================================================
% Referencial Teórico
%=======================================================================

\chapter{Theoretical Framework}

\section{Information Security}

\section{Artificial Intelligence}

\section{Attacks and Defense Mechanisms in AI Systems}


%=======================================================================
% Metodologia
%=======================================================================

\chapter{Methodology}

To comprehensively and rigorously investigate aspects related to the security of Artificial Intelligence (AI) systems, this study will adopt a Systematic Literature Review (SLR). This methodological approach, as outlined by Kitchenham \& Charters (2007) and Petersen et al. (2008), is widely used in the fields of software engineering and computer science, as it allows for the identification, evaluation, and interpretation of all available evidence relevant to a specific research question. The SLR is characterized by a systematic, explicit, and replicable process, which will ensure transparency, reliability, and scientific rigor, while minimizing bias and enabling the structured synthesis of multiple primary studies.

The methodological process to be adopted in this study will be organized into three main phases, adapted from the aforementioned guidelines: (i) Planning the Review, which will include defining the research questions, selecting data sources, constructing the search string, and formulating the inclusion and exclusion criteria; (ii) Conducting the Review, which will involve executing systematic searches in the selected databases and the subsequent screening and selection of studies; and (iii) Analysis and Synthesis of Results, which will feature the structured extraction of data and its qualitative and quantitative analysis.

\section{Research Questions}

This systematic review will be guided by the following research questions (RQs):
\begin{itemize}
    \item[\textbf{RQ1.}] What are the most investigated types of attacks on artificial intelligence systems in recent years?
    \item[\textbf{RQ2.}] What are the typical objectives of these attacks?
    \item[\textbf{RQ3.}] What types of defenses have been proposed, and how do they align with the identified attacks?
    \item[\textbf{RQ4.}] What metrics are used to evaluate attacks and defenses?
    \item[\textbf{RQ5.}] Is there a growth trend in publications on AI security in recent years?
\end{itemize}

\section{Search Strategy}

The search for relevant primary studies will be conducted in four widely recognized electronic databases, chosen for their relevance and comprehensiveness in the fields of computer science and engineering:
\begin{enumerate}
    \item IEEE Xplore – https://ieeexplore.ieee.org/
    \item ACM Digital Library – https://dl.acm.org/
    \item ScienceDirect – https://www.sciencedirect.com/
    \item Scopus – https://www.scopus.com/
\end{enumerate}

The search string will be developed by combining terms related to Artificial Intelligence with security-related terms, focusing on attacks, vulnerabilities, and defense mechanisms. The query to be used will be:
\begin{quote}
\texttt{("artificial intelligence" OR "machine learning" OR "deep learning") AND ("ai safety" OR "ai security" OR "artificial intelligence security" OR "machine learning security" OR "deep learning security" OR "adversarial attack" OR "adversarial example" OR "adversarial training" OR "backdoor attack" OR "trojan attack" OR "data leakage" OR "model inversion" OR "model stealing" OR "membership inference" OR "poisoning attack" OR "evasion attack" OR "side-channel attack" OR "privacy attack" OR "privacy-preserving machine learning" OR "robustness" OR "robust AI" OR "secure AI" OR "model hardening" OR "defense mechanism" OR "detection method" OR "attack detection")}
\end{quote}

The search will be applied only to the title and abstract fields of the articles. The searches will be restricted to studies published between 2019 and the date the search is conducted (June 2, 2025), as per inclusion criterion IC4, to ensure the currency and relevance of the findings.

\section{Inclusion and Exclusion Criteria}

\subsection{Inclusion Criteria}
Studies will be considered eligible if they meet all the following criteria:
\begin{itemize}
    \item[\textbf{IC1.}] Address security-related aspects of AI systems, including attacks, vulnerabilities, or defense mechanisms.
    \item[\textbf{IC2.}] Be a publication in a peer-reviewed scientific journal.
    \item[\textbf{IC3.}] Provide relevant contributions to at least one of the defined research questions.
    \item[\textbf{IC4.}] Published between 2019 and the date the search is conducted (June 2, 2025).
    \item[\textbf{IC5.}] Written in English.
\end{itemize}

\subsection{Exclusion Criteria}
Studies will be excluded if they meet at least one of the criteria below:
\begin{itemize}
    \item[\textbf{EC1.}] Duplicates or multiple versions of the same publication, with only the most complete and up-to-date version being retained.
    \item[\textbf{EC2.}] Opinion articles, editorials, white papers, non-peer-reviewed technical reports, conference proceedings, or other sources lacking the formal academic rigor of a journal.
    \item[\textbf{EC3.}] Secondary studies, such as systematic reviews, literature mappings, or surveys, as the focus of this review will be on primary research.
\end{itemize}

\section{Study Selection and Data Extraction Process}

The study selection process will be conducted in three stages:
\begin{enumerate}
    \item \textbf{Duplicate Removal:} After executing the searches, the results will be consolidated, and duplicates will be removed, according to criterion EC1.
    \item \textbf{Initial Screening (Title and Abstract):} The remaining studies will be evaluated based on their titles and abstracts. Those that clearly do not meet the eligibility criteria will be excluded.
    \item \textbf{Full-Text Reading:} Potentially relevant studies will be read in their entirety for final confirmation of eligibility.
\end{enumerate}

After the final selection, relevant data will be extracted and recorded in a structured spreadsheet. For each included study, information such as the type of attack investigated, attack objectives, proposed defense mechanisms, evaluation metrics, and year of publication will be collected.

The data synthesis will be performed narratively, organizing and categorizing the findings according to the central themes of the research questions (RQ1 to RQ4). For RQ5, a quantitative analysis of the temporal distribution of publications will be conducted to identify growth patterns in scientific production related to AI security.

%=======================================================================


%=======================================================================
% Cronograma
%=======================================================================

\chapter{Timeline}

This chapter presents the comprehensive timeline for conducting this systematic literature review, organized into three distinct phases with their respective activities and estimated durations. The project is strategically planned to be executed over a six-month period, from July to December 2025, ensuring adequate time for each methodological phase while maintaining the quality and rigor required for academic research.

The timeline is carefully structured to follow the systematic review methodology outlined in Chapter 3, with clear milestones and deliverables for each phase. The scheduling allows for methodical progression while incorporating necessary overlaps between activities to optimize efficiency and accommodate potential adjustments. Table~\ref{tab:timeline} presents the detailed distribution of activities across the months, providing a roadmap for proper planning and systematic monitoring of progress throughout the research period.

\begin{table}[htbp]
    \caption{Project Timeline}
    \label{tab:timeline}
    \centering%
    \begin{minipage}{\textwidth}
        \footnotesize
        \renewcommand{\arraystretch}{1.2}
        \begin{tabular}{|m{3.2cm}|m{4.3cm}|c|c|c|c|c|c|}
            \hline
            \textbf{Phase} & \textbf{Activity} & \textbf{Jul} & \textbf{Aug} & \textbf{Sep} & \textbf{Oct} & \textbf{Nov} & \textbf{Dec} \\
            \hline
            \multirow{5}{3.2cm}{\centering\textbf{Systematic Review Execution}} 
            & Database Search & X& & & & & \\
            \cline{2-8}
            & Study Selection (Titles and Abstracts) & X& & & & & \\
            \cline{2-8}
            & Full-Text Reading and Final Selection & & X& X& & & \\
            \cline{2-8}
            & Data Extraction & & X& X& & & \\
            \cline{2-8}
            & Data Analysis and Synthesis & & & X& & & \\
            \hline
            \multirow{5}{3.2cm}{\centering\textbf{Article Writing}} 
            & Introduction and Methodology Writing & & & & X& & \\
            \cline{2-8}
            & Discussion, Results and Conclusion Writing & & & X& X& & \\
            \cline{2-8}
            & Initial Revision & & & & & X& \\
            \cline{2-8}
            & Advisor Feedback & & & & & X& \\
            \cline{2-8}
            & Final Revision & & & & & X& \\
            \hline
            \multirow{4}{3.2cm}{\centering\textbf{Finalization and Defense}} 
            & Presentation Script and Material Preparation & & & & & X& \\
            \cline{2-8}
            & Advisor Feedback & & & & & X& \\
            \cline{2-8}
            & Final Adjustments & & & & & & X\\
            \cline{2-8}
            & Presentation to Committee & & & & & & X\\
            \hline
        \end{tabular}
        \\[0.5em]
        \textbf{Legend:} X  = Activity execution period\\
        \fonte{Elaborated by the author.}
    \end{minipage}
\end{table}

\section{Phase Description and Methodology Integration}

\subsection{Phase I: Systematic Review Execution (July - September)}
This foundational phase encompasses all activities related to the systematic identification, selection, and analysis of relevant literature, strictly adhering to the methodology established in Chapter 3. The phase initiates with comprehensive database searches across IEEE Xplore, ACM Digital Library, ScienceDirect, and Scopus using the predefined parameters. The systematic screening process follows a rigorous three-stage approach: duplicate removal, title/abstract screening, and full-text evaluation against inclusion/exclusion criteria. Data extraction will be conducted using standardized forms to ensure consistency and reliability of information gathering.

\subsection{Phase II: Article Writing (September - November)}  
The second phase focuses on synthesizing research findings and developing the academic manuscript through systematic analysis and interpretation of collected data. This phase includes comprehensive writing of all major sections: introduction, literature review, methodology, results, discussion, and conclusions. The writing process incorporates continuous quality assurance through iterative revisions and maintains academic rigors. Special attention is given to ensuring coherent argumentation, proper citation practices, and alignment with research objectives.

\subsection{Phase III: Finalization and Defense (November - December)}
The concluding phase encompasses all preparation activities for the final academic presentation and defense. This includes developing comprehensive presentation materials, preparing responses to potential questions, and incorporating final refinements based on advisor feedback. The phase culminates with the formal presentation to the examining committee, demonstrating mastery of the research topic and methodology.

\section{Risk Management and Contingency Planning}

The proposed timeline incorporates strategic overlaps between phases to provide flexibility and accommodate potential challenges. Should delays occur in any phase, the overlapping structure allows for schedule adjustments without compromising the overall project timeline. Additionally, regular advisor consultations throughout the process ensure continuous guidance and early identification of potential issues.

The six-month duration provides adequate buffer time for thorough execution of each phase while maintaining alignment with academic calendar requirements and institutional deadlines. This comprehensive approach ensures delivery of a high-quality systematic review that contributes meaningfully to the field of AI security research.

%=======================================================================


%=======================================================================
% Escrevendo o Texto
%=======================================================================
%\chapter{Escrevendo o Texto}

%\section{Comandos do \LaTeX}
%Como regra geral, use os comandos tradicionais do \LaTeX\ para formatar seu texto.  Neste documento procuramos demonstrar os comandos mais comumente utilizados em monografias acadêmicas.

%Neste capítulo apresentamos alguns exemplos de como colocar figuras e tabelas no seu texto.

%\section{Ilustrações}

%\subsection{Legendas}
%As legendas das figuras devem se encontrar no topo da figura e não abaixo, como usualmente colocado. Abaixo da figura, é obrigatório colocar a fonte (mesmo que a figura tenha sido do próprio autor).

%As legendas devem conter o tipo da ilustração (Figura, Tabela, etc), seguido de numeração simples (sem número do capítulo).

%Toda figura deve ser citada no texto, como nos exemplos que seguem.

%\subsection{Figuras}
%A Figura~\ref{fig:escrita} ilustra as fases psicológicas da escrita da dissertação, que também valem para monografia. Você vai se reconhecer no personagem. ;-)

%\begin{figure}
%   \caption{Fases psicológicas da escrita da dissertação}
%   \label{fig:escrita}
%   \centering%
%   \begin{minipage}{.8\textwidth}
%       \includegraphics[width=\textwidth]{images/escrita.jpg}
%       \fonte{http://www.phdcomics.com/comics/archive.php?comicid=149}
%   \end{minipage}
%\end{figure}

%\subsection{Tabelas}
%A Tabela~\ref{tab:estacoes} é um exemplo de tabela elaborada pelo(a) próprio(a) autor(a).

%\begin{table}
%   \caption{Período das estações do ano no Brasil}
%   \label{tab:estacoes}
%   \centering%
%   \begin{minipage}{.6\textwidth}
%       \begin{tabular*}{\textwidth}{ll}
%           \hline
%           \textbf{Meses} & \textbf{Estações do Ano}\\
%           \hline
%           21 de março a 21 de junho & Outono\\
%           21 de junho a 23 de setembro & Inverno\\
%           23 de setembro a 21 de dezembro & Primavera\\
%           21 de dezembro a 21 de março & Verão\\
%           \hline
%       \end{tabular*}
%       \fonte{Elaborada pela autora.}
%   \end{minipage}
%\end{table}

%\section{Resumo}
%O resumo deve conter de 100 a 500 palavras. No resumo não deve haver citações e indica-se que essa seja a última seção do texto a ser escrita. Veja abaixo uma sugestão de organização e exemplo de resumo de \cite{Moro11}.

%Sugestão (uma a três linhas para cada item):
%\begin{itemize}
%   \item Contexto geral e específico;
%   \item Questão/problema sendo investigado (propósito do trabalho);
%   \item Estado-da-arte (por que precisa de uma solução nova/melhor);
%   \item Solução (nome da proposta, metodologia básica sem detalhes, quais características respondem as questões iniciais, interpretação dos resultados, conclusões).
%\end{itemize}


%=======================================================================
% Exemplos de Citações e Referências Bibliográficas
%=======================================================================
%\chapter{Mais um capítulo}
%Texto do capítulo.

%\section{Nova seção}
%Texto da nova seção. Segundo \citeonline{inproceedings}, esta é uma forma diferente de referenciar.

%\subsection{Subseção}
%Texto da subseção, com exemplo de citação \cite{book}.

%\subsection{Subseção}
%Texto da subseção, com exemplo de citação \cite{article}.

%\section{Nova seção}
%Texto da subseção, com exemplo de citação \cite{inproceedings}.

%=======================================================================
% Referências
%=======================================================================
\bibliography{exemplo}

%=======================================================================
% Exemplo de Apêndice
% O Apêndice é utilizado para apresentar material complementar elaborado
% pelo próprio autor.  Deve seguir as mesmas regras de formatação do
% corpo principal do documento.
%=======================================================================
% \appendix
% \chapter{Informações Complementares}

% O Apêndice é o lugar para incluir textos complementares, que não são essenciais para o entendimento do assunto principal da monografia, mas que podem contribuir com informação relevante (por exemplo, uma prova matemática, uma conceituação básica, etc.).  Ele deve seguir o formato normal do documento.

%=======================================================================
% Exemplo de Anexo
% O Anexo é utilizado para a ``inclusão de materiais não elaborados pelo
% próprio autor, como cópias de artigos, manuais, folders, balancetes, etc.
% e não precisam estar em conformidade com o modelo''.
%=======================================================================
% \annex
% \chapter{Artigos Publicados}
% Existe diferença entre os Apêndices e os Anexos.  Os apêndices trazem informação escrita pelo próprio autor do trabalho, incorporando-se ao formato da monografia como um todo.  Já um anexo é um material à parte, definido/publicado por si só, e que o autor julga conveniente ser apresentado juntamente com a monografia.  Normalmente também vai apresentar formato próprio, como um artigo publicado, um folder, uma planilha, etc.
\end{document}